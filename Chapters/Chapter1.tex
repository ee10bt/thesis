\chapter{Thermal conductivity and the Earth's interior} % Main chapter title

\label{Chapter1} % Change X to a consecutive number; for referencing this chapter elsewhere, use \ref{ChapterX}

In this chapter I introduce thermal conductivity and its effects within the deep earth.

%-----------------------------------------------------------
\section{Why is \tcs important?}
%-----------------------------------------------------------
Knowledge of the thermal conductivity of solids is key in a wide range of technological applications, and for our understanding of natural systems.

%---------------------------------------
\subsection{Man-made applications}
%---------------------------------------
Low conductivities are required in thermoelectric materials, to maximise the efficiency of heat-electricity conversion \citep{Snyder2008}. Thermal conductivity determines whether a material is a conductor or insulator of heat, both of which have many technological applications.

!!! SOLAR PANELS ETC.?

%---------------------------------------
\subsection{In the context of the Earth}
%---------------------------------------
In the Earth's lower mantle thermal conductivity controls the nature of planetary convection \citep{Tosi2013}, and the heat flux out of the core which powers the geotherm. THERMAL CONDUCTIVITY WITHIN THE EARTH will be discussed further in Section~\ref{sec:ch1:cond_in_earth}.

Within the lower mantle, thermal conductivity influences the rate at which heat is transferred from core cooling towards the surface, and more importantly the mechanisms by which it does so \citep{Lay2008}. High thermal conductivity systems will preferentially transport heat by conduction. Systems will convect (ADVECT?) where there is too much heat to be transpoted by conduction alone (i.e. low conductivity conditions).

Observations of plume structures and cyclic subduction patterns \citep[see][]{Garnero2008} suggest convective behaviour in the lower mantle. Thermal conductivity is poorly constrained in this region, obtaining a comprehensive depth profile is not a trivial task. Additional difficulties are pressure, temperature, and compositional dependences, including isotopic variation \citep{Tang2010,Dalton2013,Tang2014} and inclusion of impurities \citep{Manthilake2011,Ammann2014,Ohta2014}.
















%-----------------------------------------------------------
\section{What is \tc?}
%-----------------------------------------------------------
\label{sec:what_is_tc}

Thermal conductivity is a material property, indicative of the ease with which heat is transferred through conduction. For known substances thermal conductivity spans about six orders of magnitude, from silica aerogels with 0.005~\wmks \citep{Lee1995} to graphene with 5000~\wmks \citep{Balandin2008}.

The transfer of thermal energy can occur between an object and its surroundings, two bodies brought into contact, or along a temperature gradient within an object. The possible mechanisms by which this can occur are conduction, convection, and radiation. Conduction is the transfer of heat by atomic vibrations, and electron transport in metallic substances (such as in the outer core). Convection is the motion of thermal energy via a moving medium, generally in liquids and gases (but expected in the mantle). Density differences are the driving force for convection, due to the volume change associated with thermal expansion. Radiation refers to the transport of heat by electromagnetic radiation in the form of photons.

I am primarily concerned with the atomic vibrations of lattice conduction through the lower mantle, and secondly the convective behaviour therein. Mantle minerals are expected to be insulators, meaning there is no contribution from electron transport. The radiative component of thermal conductivity in the mantle is thought to be small \citep{Goncharov2008}, and will not be determined as part of this work. In the event that radiation contributes significantly to the effective thermal conductivity \citep{Keppler2008}, it can simply be added to the lattice component.

Convection in them mantle is dependent on the \tc, and will occur if heat cannot be sufficiently transported by conductive processes. The value of the Rayleigh number describes the behaviour of heat flow in a fluid, relative to a critical value for the system. Many variables are required to determine the Rayleigh number, but most importantly here it is inversely proportional to \tc. If the calculated Rayleigh number is lower than the critcal value, conduction is the dominant process. When the Rayleigh number is greater, the ratio between it and the critcal value decribes the vigour and style of convection.

%---------------------------------------
\subsection{Mechanisms of heat transport}
%---------------------------------------

%-------------------
\subsubsection{Electron}
%-------------------

Close parallels exist between thermal and electrical conduction. The conduction of thermal energy in metals is predominantly due to the motion and interaction of free electrons. Heat is transferred as electrons move and collide in the lattice. There is no net transport of electrons in order to maintain charge neutrality within the lattice.

%-------------------
\subsubsection{Photon}
%-------------------

Any body with a non-absolute zero temperature emits thermal radiation as light, or photons. Radiative thermal conductivity is determined by a material's optical absorptivity, which describes how heat is transferred by electromagnetic radiation within a material. On the atomic scale, charged particles vibrate and emit light. Energy is transferred from one particle to another when this light is scattered or absorbed. The transfer of heat by radiation is limited similarly to the transfer of visible light, with difficulty passing through opaque media. Unlike lattice conductivity at mantle conditions, radiative conductivity increases with temperature \citep{Hofmeister1999}. This relation has been used to assume thermal conductivity could be constant through the lower mantle if radiative processes are significant, where the lattice component decreases at the same rate the radiative increases \citep{Tang2014}.

%-------------------
\subsubsection{Phonon}
%-------------------

The final way heat is conducted is by the vibration of atoms, where the velocity of an atom is proportional to its heat. As an atom vibrates, forces act on neighbouring atoms, inducing motion. Hot atoms in a closed system transfer energy to cold atoms in this way, until energy and temperature are both equilibrated. Considering a crystalline arrangement of atoms, there is long-range structure and well-defined bonds between atoms. Much like standing waves on a string, atoms can vibrate in-phase. These patterns of vibrations are called phonons, and can be differentiated by wavelength and the relative motions of atoms.

Similar to the observed behaviour of photons, phonons exhibit wave-particle duality. This is necessary to describe how phonons interact with structural discontinuities and in a phonon-phonon collision. As a particle, a phonon is a quantised packet of vibrational energy. Unlike electronic heat conduction, there is a net motion of phonon ``particles'' from hot to cold regions, and the amount of active phonon modes increases with temperature. For a given material at constant conditions, there is a temperature at which thermal conductivity peaks. Up to this temperature, some phonon wavelengths are inactive in heat transport. Above this temperature, increased frequency of phonon-phonon collisions cause scattering, tending to reduce the efficiency of heat transfer. In the lower mantle, we are above this critical temperature for all physical pressure conditions, thus increasing temperature reduces conductivity.





%---------------------------------------
\subsection{What affects it?}
%---------------------------------------
pressure, temperature, and composition

temperature dependence from 0~K

pressure increase

refer to composition effects in chap 4

Heat is transported as lattice virbrations, or phonons. The further phonons travel before scattering (mean free path, MFP), the more efficient the heat transport and thus higher the thermal condutivity. A number of effects (MENTION MATTHIESSEN'S RULE) cause phonons to scatter: (1) collisions with other phonons in the lattice, (2) boundaries or defects in the material, and (3) impurities in the atomic structure. 






%-----------------------------------------------------------
\section{Previous work - geophysics}
%-----------------------------------------------------------

!!! EXPLAIN THE EARTH HERE

The lower mantle encompasses the region between the mantle transition zone (660~km deep, $\sim$1900~K, $\sim$25~GPa) and the CMB (2891~km deep, $\sim$4000~K, $\sim$136~GPa). The composition of this region can be approximated as 75\% bridgmanite (MgSiO$_3$, magnesium silicate perovskite), 20\% periclase (MgO, magnesium oxide), and  5\% calcium silicate (CaSiO$_3$) perovskite. All of which are insulators and past their Debye temperatures at lower mantle conditions, with the potential for the inclusion of impurities such as iron and aluminium.

A popular opinion is that bridgmanite is stable in the lower mantle until the bottom few 100~km, where it undergoes a pressure-driven phase change to post-perovskite \citep{Murakami2004,Oganov2004}. In places, close proximity to the CMB might transform post-perovskite back to perovskite structure due to the increased temperature. This ``double crossing'' of the bridgmanite stability range can be imaged seismically, where lens-like bands of post-perovskite are shown to pinch out laterally \citep{Lay2006}.

A range of atomic scale simulation methods are available to determine the lattice thermal conductivity of materials. These are invaluable for calculating thermal conductivity at conditions of which experiments are difficult, e.g. the extreme conditions found in the Earth's lower mantle (pressures and temperatures up to 136~GPa and 4000~K at the core-mantle boundary). 

%---------------------------------------
\subsection{STUFF THAT IS INFLUENCED BY CONDUCTIVITY}
%---------------------------------------
\label{sec:ch1:cond_in_earth}

Thermal conductivity in the deep Earth influences dynamic processes such as mantle convection and heat loss from the core \citep{Lay2008}. In this section I will discuss the prominent thermal conductivity-dependent processes.

%-------------------
\subsubsection{Mantle dynamics}
%-------------------

In the lower mantle \tcs changes with pressure, temperature, and composition, influencing features on a large scale. For example, \citet{Naliboff2006} used numerical models of mantle convection to show size and stability of convective plumes are sensitive to thermal conductivity above the core mantle boundary (CMB).

\citet{Dubuffet2000} investigated the effects of temperature and pressure-dependent \tcs on mantle convection, finding that depth-dependent \tcs encouraged heat transport via convective plumes. Compared to a constant conductivity model, vertical heat transfer was concentrated to these ``pipe-like'' structures, despite the horizontally-averaged heat flow for both systems being around the same value. Variable conductivity, even in one dimension, increased the spatial and temporal stability of convection. Plumes were thicker, had heads of larger surface area, and were hotter, compared to the uniform conductivity mantle model.

%-------------------
\subsubsection{Heat flow}
%-------------------

The most accessible estimate of the Earth's energy is the total heat flow at the surface, of which a value of $46\pm3$~TW is accepted as the upper bound. Sources of surface heat flow include; radiogenic heating ($20\pm3$~TW), mantle cooling (8--28~TW), and the conduction of heat across the CMB from core cooling \citep{Lay2008}. Conductive heat flow is constrained by thermal conductivity, a model of which is not available for all Earth conditions.

Better constraints on thermal conductivity are required to estimate CMB heat flow. This in turn would tell us more about the temperatures either side of the CMB, as well as the presence and nature of the lower mantle thermal boundary layer (TBL). Employing the most commonly used value for lower mantle conductivity, 10~\wmk~\citep{Lay2008}, heat flow across the CMB is expected to be 5--13~TW~\citep{Lay2008}. The value of 10~\wmks used by \citet{Lay2006} is an estimate of lowermost mantle \tc, based on extrapolation of a measurement at ambient conditions \citep{Osako1991}. Both higher and lower values have been proposed \citep[4--16~\wmk,][]{Manthilake2011}, illustrating how poorly constrained thermal conductivity is at CMB-relevant pressure/temperature conditions.

%-------------------
\subsubsection{Geomagnetism}
%-------------------

Using shear wave velocity as a proxy for CMB heat flow, \citet{Gubbins2007} showed that variations in mantle temperature gradients above the CMB can influence Earth's geodynamo. The present day magnetic field at the surface has four lobes, and these align above regions of fast shear wave velocity on the CMB. Ignoring compositional effects in the mantle, seismically-fast regions can be assumed to be cold. Considering Fourier's law (CITE), colder regions facilitate larger heat flows through steeper temperature gradients from the isothermal CMB.

\citet{Gubbins2007} recreate the geomagnetic observation of the aforementioned lobes using a core geodynamo simulation, where the upper boundary (CMB, lowermost mantle bottom) condition was a laterally varying heat flux. Knowing the \tc, especially as it changes with temperature, would better constrain mantle boundary conditions used in this and similar core dynamics models \citep{Ammann2014}.




%---------------------------------------
\subsection{DETERMINATIONS OF TC FOR EARTH MATERIALS/CONDITIONS}
%---------------------------------------
Many studies assume lowermost mantle thermal conductivity to be 10~\wmk~\citep[e.g.][]{Lay2008}, but uncertainty in the extrapolation of experimental results made at low pressures and temperatures gives a range of 4--16 \wmk~\citep{Brown1986, Osako1991, Hofmeister1999, Goncharov2009, Manthilake2011, Ohta2012}. I give a review of experimental and computational determinations of conductivity for Earth-relevant conditions, where the Direct and Green-Kubo calculation methods are elaborated later on in Section REF (somewhere in chap2).

%-------------------
\subsubsection{Experiments}
%-------------------

There have been several computational studies to calculate the lattice thermal conductivity of bridgmanite at CMB conditions. \citet{Osako1991} measured the lattice thermal conductivity of MgSiO$_3$ perovskite, using a modified \AA ngstrom method. They investigated a temperature range of 160--340~K at ambient pressure. At 300~K, a conductivity of 5.1~\wmks was obtained. This value is similar to that reported for chemical and structural analogues, MgSiO$_3$ enstatite (5.0~\wmk~REF) and CaTiO$_{3}$ perovskite (4~\wmk~REF). The authors extrapolated the value to mantle conditions, neglecting radiative thermal conductivity. They predicted a value of 3.0~\wmks just beneath the mantle transition zone at 1900~K, and 12.0~\wmks at the top of the \ddds layer at 2500~K, a four-fold increase. Thermal conductivity is highlighted as an important indictor of lowermost mantle structure, whether or not the \ddds layer can behave as a thermal boundary between core and mantle.

\citet{Manthilake2011} measured MgSiO$_3$ perovskite at 26~GPa and 473--1073~K, and periclase at 8 and 14~GPa between 373--1273~K. In order to estimate values of thermal conductivity at the top and bottom of D$^{\prime \prime}$ for a lower mantle compositional model of 4~perovskite~:~1~periclase, the authors extrapolated their measurements to high temperature and pressure. For an iron-free mantle, thermal conductivities of $18.9\pm1.6~$\wmks and $15.4\pm1.4$~\wmks are estimated for the top of D$^{\prime \prime}$ and CMB respectively. Similarly, for a mantle composition with Fe, thermal conductivities of $9.1\pm1.2$~\wmks and $8.4\pm1.2$~\wmks are calculated. This highlights the importance of impurities in controlling thermal conductivity in the lower mantle.

\citet{Ohta2012} measured the lattice thermal diffusivity of MgSiO$_3$ perovskite and post-perovskite at room temperature and pressures up to 144~GPa (using a diamond-anvil cell and light heating thermoreflectance). These results suggest a majority perovskite lowermost mantle would have conductivity of $\sim$11~\wmk, and that parts of the lowermost mantle where post-perovskite is stable will have a conductivity approximately 60\% higher. The authors suggest that these differences in conductivity between phases will not have a large effect on CMB heat flux, assuming the double-crossing perovskite phase model. The lattice conductivity of \mgsios perovskite is shown to increase with pressure and decrease with temperature as expected. The inclusion of impurities is expected to decrease lattice thermal conductivity.

MANTHILAKE'S HIGH T RESULTS? CAN'T FIND REFERENCE?

%-------------------
\subsubsection{Calculations}
%-------------------

%My approach is similar to that of \citet{Ammann2014}, who use the direct method and interatomic potentials reporting a value of $\sim$8.5~\wmk. \citet{Stackhouse2015} again use the direct method but with density functional theory, yielding conductivity of $6.8\pm0.9$~\wmk. Using Green-Kubo, \citet{Haigis2013} report a value of $12.4\pm2.0$~\wmks for conditions of 3000~K, 139~GPa. \citet{Tang2014} and \citet{Dekura2013} employed first principles, anharmonic lattice dynamics techniques, obtaining values of $\sim$1~\wmks (CMB conditions) and 2.3~\wmks (for 4000~K and 100~GPa) respectively. These results are much lower than other studies, and could be because of LD TRUNCATION OF CONDUCTIVITY [CRITICAL ANALYSIS].

\citet{Haigis2012} used the Green-Kubo method (refer to REF) to calculate the lattice thermal conductivity of bridgmanite, post-perovskite, and periclase at lower mantle conditions. Assuming an iron-free composition with four parts bridgmanite to one part periclase, a model is constructed of density and temperature-dependent thermal conductivity along a geotherm. This model suggests great variation over the lower mantle, with a value of 9.5~\wmks at the top and 20.5~\wmks above \ddd. Based on the results of \citet{Manthilake2011}, \citeauthor{Haigis2012} suggest the inclusion of iron will lower thermal conductivity by up to half, bringing their result in line with \citet{Lay2006} ((10 WMK???)). The authors estimate the CMB heat flux to be 10.8~TW for an iron-bearing perovskite/periclase aggregate, dropping slightly to 10.6~TW for a similar post-perovskite aggregate. These values match other predictions of CMB heat flux \citep[e.g.][]{Lay2008}.

\citet{Dekura2013} used \ais anharmonic lattice dynamics with density functional theory (DFT) to calculate the lattice thermal conductivity of bridgmanite. At temperature of 300~K, they found conductivity increases from 9.8~\wmks at 23.5~GPa to 43.6~\wmks at 136~GPa. Temperature-dependence was found to be negative at 100~GPa, where conductivity decreases from 28.1~\wmks at 300~K to 2.3~\wmks at 4000~K. The pressure and temperature conditions used cover the entire range of accepted lower mantle conditions. From their results they calculated a Rayleigh number ($Ra$) of 10$^5$--10$^7$ for the mantle, in agreement with the geophysically-expected thermal convection (when $Ra < 10^3$--$10^4$). These results suggest that a CMB region at 136~GPa and 3200~K will have a conductivity of 5.3~\wmk, corresponding to a heat flux of 3--6~TW.

\citet{Ammann2014} used the direct method, a non-equilibrium molecular dynamics technique, to calculate the lattice thermal conductivity of bridgmanite and \ppvs under \ddds conditions. They found the conductivity of \ppv to be around 50\% larger than \bdg for the same conditions (12~\wmks compared to 8.5~\wmk). This relation is true even in the TBL, where increases in temperature reduce lattice conductivity for all \mgsios phases. An interesting result of their work is the observation of anisotropy for \ppvs conductivity. This may lead to a feedback mechanism, influence the formation and stability of convective plume structures.

\citeauthor{Ammann2014} also investigated the effects of impurities on conductivity, substituting magnesium with iron. They increase the mass of Mg atoms to resemble Fe, which tends to reduce conductivity. The lower mantle distribution of iron is not yet well-understood, specifically the partitioning between \bdg, \ppv, and periclase. Interestingly, the authors observed saturation in the conductivity reduction associated with atomic impurities, even for small Fe concentrations (((FIG))). Extrapolations of variable-composition experimental results must be applied carefully, increasing iron content past a certain point will not reduce conductivity any further.

\citet{Tang2014} %performed first-principles calculations to assess the \tcs of bridgmanite and periclase at lower mantle pressures and temperatures. To construct a mantle model, they consider the effect of iron impurities 















%-------------------
\subsubsection{Radiative conductivity}
%-------------------

\citet{Hofmeister1999} produced a model of thermal conductivity for the entire mantle using data from infrared reflectivity methods. The radiative component at maximum was found to be small compared to the lattice conductivity, between 10--15\% depending on the geotherm model used. This corresponds to radiative conductivity values of 0.67--0.82~\wmks compared to 5.8--6.7~\wmks for the total conductivity at the top of \ddd.

\citet{Keppler2008} studied the near-infrared and optical absorption spectra of silicate perovskite up to pressures of 125~GPa at room temperature. From both their tests and visual inspection, it can be shown that their synthesised perovskite remains transparent at high pressures. Extrapolating their results to high temperatures (4500~K) they suggest that the maximum radiative thermal conductivity above the CMB is around 10~\wmk, implying that radiative conductivity is likely to be an important component of the total conductivity at lower mantle conditions. The study does not measure the variation of absorption spectra with temperature and pressure, of which experimentation is currently unfeasible.

\citet{Goncharov2008} performed a similar optical absorption spectra analysis up to 133~GPa, but with the opposite conclusion to \citet{Keppler2008}. They agreed the radiative conductivity was dependent on the amount, redox state, and spin state of iron, but disagreed with its significance. \citet{Goncharov2008} estimated radiative conductivity would not exceed 0.54~\wmks at the top of the \ddds layer, a value in line with \citet{Hofmeister1999} and at odds with \citet{Keppler2008}. Until better constrained, it is convenient to assume the radiative component is small compared to the lattice component. As stated earlier (Section~\ref{sec:what_is_tc}), the two components can simply be added to determine the total thermal conductivity (for the lower mantle).

\citet{Tang2014} re-evaluated the works of \citet{Keppler2008} and \citet{Goncharov2008} to create a profile of radiative conductivity in the lower mantle. \citeauthor{Tang2014}'s profile suggests that the previous works have a more reasonable agreement than they show, using analysis which gives an upper bound on the conductivity. Radiative heat transfer is inhibited in the same way as conductive, by impurities and grain boundaries which are not considered when calculating this upper bound. Unlike lattice thermal conductivity, radiative conductivity increases with temperature, steeply so in the mantle thermal boundary layer. When the opposing temperature dependencies of lattice and radiative conductivity are considered in tandem, they suggest that the thermal conductivity of the lower mantle is largely temperature-independent above the \ddds region at around 3~\wmk. Thermal conductivity increases to 5.5 \wmks in the TBL, due to the increased significance of the radiative component.








%-----------------------------------------------------------
\section{Thesis outline}
%-----------------------------------------------------------
In Chapter \ref{Chapter2} we provide an overview of the methods and expand on issues. I outline my computational approaches, for the non-equilibrium molecular dynamics direct method and equilibrium molecular dynamics Green-Kubo method. I show convergence of computed conductivity with respect to simulation cell size and shape 

In Chapter \ref{Chapter3}, PRESSURE/TEMPERATURE EFFECTS. DISCUSS [P/T] SCALING LAW / THEORETICAL MODEL

In Chapter \ref{Chapter4}, ADAPTING SYSTEM TO INCLUDE IRON, DETERMINING EFFECT OF IMPURITY CONTENT, PRODUCING A MODEL OF THE LOWER MANTLE CONDUCTIVITY WITH VARIABLE TEMPERATURE AND BRIDGMANITE COMPOSITION

%---------------------------------------
\subsection{Aims}
%---------------------------------------

%---------------------------------------
\subsection{Objectives}
%---------------------------------------


