% Chapter Template

\chapter{Using STUFF to do THINGS} % Main chapter title

\label{Chapter5} % Change X to a consecutive number; for referencing this chapter elsewhere, use \ref{ChapterX}

%-----------------------------------------------------------
\section{Main Section 1}
%-----------------------------------------------------------

Lorem ipsum dolor sit amet, consectetur adipiscing elit. Aliquam ultricies lacinia euismod. Nam tempus risus in dolor rhoncus in interdum enim tincidunt. Donec vel nunc neque. In condimentum ullamcorper quam non consequat. Fusce sagittis tempor feugiat. Fusce magna erat, molestie eu convallis ut, tempus sed arcu. Quisque molestie, ante a tincidunt ullamcorper, sapien enim dignissim lacus, in semper nibh erat lobortis purus. Integer dapibus ligula ac risus convallis pellentesque.

%---------------------------------------
\subsection{Subsection 1}
%---------------------------------------

Nunc posuere quam at lectus tristique eu ultrices augue venenatis. Vestibulum ante ipsum primis in faucibus orci luctus et ultrices posuere cubilia Curae; Aliquam erat volutpat. Vivamus sodales tortor eget quam adipiscing in vulputate ante ullamcorper. Sed eros ante, lacinia et sollicitudin et, aliquam sit amet augue. In hac habitasse platea dictumst.




I am able to construct a model of lower mantle heat flux, including the effect of LLSVP properties. Using LEMA, I specify a CMB base condition, and a TBL condition at some height above it. In all models I consider, the CMB will be isothermal. The TBL will have a variable mean temperature, and regions of higher and lower temperature. The lower mantle has an average Fe\%, but the LLSVPs are enriched compared the depleted surrounding bulk lower mantle. By varying the temperature and composition at the CMB, conductivity can be altered using the model from PREVIOUS-REF. The difference in temperature between CMB and TBL, and the undulations on the latter, control the lower mantle temperature gradient. Heat flux can be calculated using conductivity and temperature gradient via FOURIER'S LAW. While the magnitude of the heat flux will change with parameters, perhaps more interesting is the lateral variation of this parameter. This will show how the lower mantle heat flux is sensitive to the conditions therein, and what condition or combination of conditions is most significant. The results from these calculations can then be compared to observables, dismissing scenarios which are unfeasible or unstable. There are many variables in the lower mantle, and observables to compare to. The variable conductivities from this work will play an important role in constraining CMB heat flux.


%VARIABLES TO PLAY WITH

%Temp_cmb
%Temp_tbl
%Temp_diff
%Temp_hot
%Temp_cold

%Height_tbl
%Height_llsvp (peak height, how is shape considered, sph. harm.?)

%Comp_mean
%Comp_lm
%Comp_llsvp
