% Chapter Template

\chapter{Constraining the finite-size effects of molecular dynamics methods to compute thermal coductivity} % Main chapter title

\label{Chapter3} % Change X to a consecutive number; for referencing this chapter elsewhere, use \ref{ChapterX}

%-----------------------------------------------------------
\section{\label{sec:3.intro}Introduction}
%-----------------------------------------------------------

Knowledge of the thermal conductivity of solids is key in a wide range of technological applications and for our understanding of natural systems. For example, in the Earth's lower mantle thermal conductivity controls the nature of planetary convection \citep{Tosi2013}, and the heat flux out of the core which powers the geotherm. Low thermal conductivities are required in thermoelectric materials, to maximise the efficiency of heat-electricity conversion \citep{Snyder2008}.

A range of atomic scale simulation methods are available to determine the lattice thermal conductivity of materials. These are invaluable for calculating thermal conductivity at conditions of which experiments are difficult, e.g. the extreme conditions found in the Earth's lower mantle (pressures and temperatures up to 136~GPa and 4000~K at the core-mantle boundary).

(MOVE - to where though?) Many studies assume lowermost mantle thermal conductivity to be 10~\wmk~\citep[e.g.][]{Lay2008}, but uncertainty in the extrapolation of results made at low pressures and temperatures gives a range of 4~-~16 \wmk~\citep{Brown1986, Osako1991, Hofmeister1999, Goncharov2009, Manthilake2011, Ohta2012}.

%---------------------------------------
\subsection{\label{sec:3.FSE}Finite-size effects}
%---------------------------------------

THE PROBLEM OF FSE IN COMPUTATIONAL TECHNIQUES

EXACERBATED BY RANGE OF LOWER MANTLE CONDITIONS

The effect of FSE on conductivity results depends on the magnitude of conductivity/phonon MFP/physical conditions. At low kappa/low MFP/high T/ (my 4000~K), no BPT is observed, and short cells (>16 unit cells) can be used for extrapolation. In fact, short cells must be used to extrapolate, unless CSA considersations are made to ensure convergence of long cell results.

At high kappa/high MFP/low T (my 1000~K), BPT must considered at the shorter cells (just 6?). Effectively there is a "sweet-spot", a window of cell lengths for a given CSA that produce consistently-converged results. Long cells outside of the window require a larger CSA, short cells outside show BPT. At 4000~K the lower limit of the window is smaller than the minimum cell length considered, and the upper limit is between 16-24 unit cells. For 1000~K the lower limit of the window moves inside the simulated cell length range around 6-8 unit cells (OR MORE?). The upper limit of the window appears to be larger than 96 unit cells, including all long cells up to this value produces an extrapolation in agreement with GK.

%-----------------------------------------------------------
\section{\label{sec:3.methods}Computational approaches}
%-----------------------------------------------------------

In this section I talk about my specific approaches to applying the methods discussed in Chapter \ref{Chapter2}

%---------------------------------------
\subsection{\label{sec:3.direct}Direct method}
%---------------------------------------

The simulation supercell is split into sections along its length, each half a unit cell wide. 
Two of these sections, half the supercell length apart, are designated as the heat source and heat sink. We measure (HOW?) the temperature in all sections to obtain the temperature gradient. Heat flows in both directions from the hot section because of cell periodicity [NEED DIRECT METHOD DIAGRAM BY THIS PARA], meaning there are two temperature gradients to average. Where L is supercell length in unit cells and S (= 2L) gives the number of sections, we obtain S/2 + 1 temperature points to fit the gradient. Because the temperature gradient is non-linear around the heat source and sink, we ignore S/12 sections (rounded to nearest integer) from both ends of the temperature gradient. For a given simulation cell we fit S/3 + 1 points to obtain the temperature gradient. We use a minimum supercell length of 6 unit cells (12 sections, 5 data points), in order for sufficient fitting of the temperature gradient. [HOW NECESSARY IS THIS PARA, VERY JARGONY]

(MOVE TO RESULTS?) Changing the width of the heated sections has no effect on the conductivity result. Furthermore, changing the width (and thus number) of temperature bins has no effect on the sampled gradient, assuming resolution is large enough to capture the non-linear region around the heat source/sink.

An important factor for utilising the direct method is maintaining a sensible temperature gradient where Fourier's law remains valid, i.e. conductivity is constant along the length of the cell. Thermal conductivity is strongly temperature-dependent at upper lower-mantle conditions (1000~K), it is therefore undesirable to have substanially different conductivities as a of function of temperature across the cell. The opposite case is also true, the difference in temperature between hot and cold sections must be larger than the uncertainty in the average system temperature. 

We typically observe fluctuations in temperature of around $\pm$50~K during temperature equilibration, and look for temperature increases/decreases on the order of 10\% the mean temperature. We control the magnitude of the gradient by altering the interval at which heat is exchanged. To produce the desired gradients we find shorter intervals are required as cell length decreases, cross-sectional area increases, and system equilibrium temperature decreases. 

%---------------------------------------
\subsection{\label{sec:3.GK}Green-Kubo}
%---------------------------------------

The bridgmanite unit cell does not have cell dimensions resembling a cube (a:b:c = 1:1:1.4), so we use supercell structures of 3x3x2, 4x4x3, 5x5x4, 6x6x4 etc. to make an approximately cubic simulation cell (((WHY IS IT CUBIC THOUGH I HAVEN'T CONVINCED MYSELF))). Temperature initialisation (NVT) of 1~ns is run to ensure convergence of system pressure and temperature. To obtain heat flux auto-correlation functions, a simulation for each initial temperature condition is run for X~ns, with 9 successive repeats for a total of 10 jobs. This gives 10 ACFs from each initial condition. Simulation runs are split in this manner to be feasible computationally, jobs submitted to the high-performance computing facilities have a maximum length of 48~hours. Each job finished in this manner produces an ACF, somewhat of a bootstrapping process on the total simulation series.




3x3x2 - X = 10~ns, for 20 inital conditions - 2~$\mu$s total time

4x4x3 - X = 10~ns, for 30 inital conditions - 3~$\mu$s total time

5x5x4 - X = 5~ns, for 20 inital conditions - 1~$\mu$s, X = 1~ns, for 70 inital conditions - 0.7~$\mu$s, 1.7~$\mu$s total time

6x6x4 - X = 1~ns, for 80 inital conditions - 0.8~$\mu$s total time

THIS INFO IN A TABLE, OR JUST GIVE FOR THE RELEVANT VOLUME?

I DON'T LIKE THE INCLUSION OF ALL THE ABOVE INFORMATION

In this study we compute ACFs up to correlation lengths of 100~ps, with (100,000) 1~fs timesteps. This length is longer than required but selected as a proof of concept to show convergence in the conductivity result, additionally to display the extent and behaviour of drift in the integrals for long correlation times. We show in Figure \ref{fig:acf_decay} that the magnitude of the ACF decays to much less than 1\% of its initial value around a correlation time of 2~ps, inferring the start of convergence for the integral and thus conductivity.

ACFs produced by each simulation are integrated seperately, and averaged into a single series. This process is performed for heat fluxes in each crystallographic direction, to allow analysis of anisotropy and finite system size effects.  From this combined integral we pick a window of correlation lengths to capture a flat, converged region (or the section just after the 'bottleneck' if convergence is not obvious). This correlation length window is then applied to all integrals constituent to the combined series, giving a sample integral averages and corresponding standard deviations. A weighted average is then taken of these data points, to give a single value with uncertainty. This value is directly proportional to thermal conductivity, as given by Equation \ref{gk-int}. REFERENCING TOO FAR BACK? REPEAT EQUATION?

Considering bridgemanite at lower mantle conditions, we find correlation time windows in the range of 2-30~ps to be suitable. At the low-end, this allows the initial high-variability in integral value to be ignored. At the high-end, the time is long enough for good sampling of the integral, but short enough to ignore the drift-effects. The magnitude and range of the window typically increases with conductivity (or with decreasing temperature etc.), e.g. 2-10~ps at 4000~K, and 10-30~ps at 1000~K.

%-----------------------------------------------------------
\section{\label{sec:3.results}Results}
%-----------------------------------------------------------

%---------------------------------------
\subsection{\label{sec:3.GK.results}Green-Kubo}
%---------------------------------------
A supercell volume of 3x3x2 (((REFER TO GRAPH, WILL NEED 1000K TOO))) fails to reproduce conductivities on the same order as the larger cells for all directions. We identify 4x4x3 and larger cells as being converged with respect to cell volume (((PROBLEMATIC STATEMENT, NOT CONVINCING, BY WHAT METRIC?))). This a useful result in terms of computation efficiency, as 6x6x4 supercells are 3 times as large (VOLUMOUS? REFERENCE ATOM COUNT?) as 4x4x3.

%---------------------------------------
\subsection{\label{sec:3.direct.results}Direct method}
%---------------------------------------
When determining finite-size effects, it is important to consider the scenario with largest phonon mean-free paths. Phonon MFPs are largest at low temperatures (beyond the Debye temperature) and high pressures. In light of this we consider pressure of 136~GPa and temperatures of 1000~K and 4000~K. 136~GPa / 4000~K represents the expected conditions of the core-mantle boundary, whereas 136~GPa / 1000~K is unphysical in the context of the Earth but maximises MFP.  UPDATE - GK RUINS EVERYTHING

By computing conductivities across a range of cell lengths we show that direct method simulations with small cross-sectional areas fail to produce converged results with respect to larger CSAs. Without considering any extrapolation, it is clear that small CSA cells overestimate conductivity (Figures \ref{fig:direct_length_graph_4000} \& \ref{fig:direct_length_graph_1000}) at conditions of both 1000~K and 4000~K. On both figures the results for cells with CSA 2x2 and larger plot close to on top of each other. Producing the same results as 8x8 cells, we conclude that cells with CSA of 2x2 are suitable for direct method simulations of bridgmanite.

Now considering CSAs of 2x2, we examine the divergence of conductivity result with cell length from an expected linear trend (Figure \ref{fig:direct_length_graph}). As mentioned in Section \ref{sec:theory.direct}, cells that are too long or short cause a conductivity result to be overestimated. At 4000~K we find that cells up to 24 unit cells length ($<$0.06 in inverse length) produce a reasonably linear trend. At this condition, there is no reasonable overestimation due to short cells and ballistic phonon transport. However at 1000~K where the MFP is longer, a cell of length 6 unit cells (inverse length 0.167) produces conductivity larger than expected. The same long-cell divergence is found $>$24 unit cells length, the onset of insufficient phonon sampling. For all direct method simulations of bridgmanite at lower mantle conditions, we recommend employing cell lengths of 8~-~$<$24 unit cells. Due to the increasing computational cost associated with cell length (especially for ab initio methods), we recommend the longest cells be 16 unit cells. FINITE SIZE EFFECTS INCREASE WITH MFP/KAPPA

Now we consider CSA of 2x2 and supercell lengths of 8, 10, 12, and 16 unit cells for comparison with the results from Green-Kubo. After performing a weight least squares regression (extrapolation) on the direct method results, we obtain a conductivity with uncertainty. Figure \ref{fig:gk-direct} shows the extrapolation and Green-Kubo result (at x~=~0). They agree within error, meaning we have chosen a suitable set of criteria for working with direct method results. DO I NEED TO PROVE THAT OTHER LENGTHS/EXTRAPOLATIONS DON'T MATCH GK? FINE AT HIGH T / SHORT MFP, LOW T / HIGH MFP NEEDS LONGER CELLS TO EXTRAPOLATE, OR MORE SHORTER CELLS IGNORED.



%-----------------------------------------------------------
\section{\label{sec:3.Summary}Summary}
%-----------------------------------------------------------

For bridgmanite (at conditions representing the lower mantle), we show that use of the direct method for calculation of thermal conductivity will lead to an overestimate if the simulation cell is too long (\textgreater 16 unit cells, 4000 ONLY!!!). Small cross-sectional areas (\textless 2x2 unit cells) also overestimate the thermal conductivity. This informs future work using Density Functional Theory, and will allow a model of lower mantle conductivity considering composition to be established [[[OOPS]]].

(ASSUMING THE RESULTS ARE CORRECT AND AGREE WITH GK) We see the non-linear region as described by \citet{Sellan2010} for the cell length of 6 unit cells at 1000~K, which has individually higher conductivity than expected from the linear fit through data points corresponding to lengths of 8-16 unit cells. When included in the extrapolation, this reduces the gradient of the fit, raising the intercept and thus causing conductivity to be underestimated. At temperature of 4000~K, the 6 length cell is inline with the fit through other cells with length less than 16 unit cells. As the ratio of cell length to phonon MFP increases with temperature, we believe the onset of divergence as described by Sellan et al. moves to the right (??? - MENTION ACTUAL EFFECT - QUANTIFY RATHER THAN REFERENCING GRAPH). A shorter MFP needs shorter cell lengths to display divergent conductivity, of which we have not sampled (at high temperature). DOING THE DIRECT METHOD WITH CELLS OF LENGTH LESS THAN 6 UNIT CELLS AT ANY TEMPERATURE IS A BAD IDEA BECAUSE ... 

We find conductivity is definitely dependent on CSA, but we were not able to increase CSA enough to eliminate aspect ratio-dependent divergence as reported by \citet{Hu2011}. This does support our conclusion ignoring long cell lengths however, in order to keep the aspect ratio within a reasonable limit and ensure a linear fit is extrapolated. (EVEN THOUGH 48x8x8 HAS A SMALLER RATIO than 8x2x2?)

(WAFFLE ALERT) Ignoring the specifics of this study, we stress the importance of performing finite-size analysis when performing direct method calculations.  Direct method cells spanning a range of lengths must be considered to find the linear regime for extrapolation. Cross-sectional area must be increased until the conductivity result converges. The same can be said about the Green--Kubo method, where the result converges with increasing volume. These effects vary with phonon mean-free path, sensitive to pressure, temperature, and compositional variations such as impurities. Completing finite-size effect analysis at conditions with the largest phonon mean-free path / thermal conductivity ensure all other conditions represent converged results. We believe classical molecular dynamics with interatomic potentials to be an excellent way of quantifying these effects quickly, performing ab initio methods (SHOULD I TAKE THIS SENTENCE OUT, NO PROOF OF THIS CLAIM).
