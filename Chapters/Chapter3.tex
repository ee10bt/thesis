% Chapter Template

\chapter{The Chapter formerly known as Paper 1} % Main chapter title

\label{Chapter3} % Change X to a consecutive number; for referencing this chapter elsewhere, use \ref{ChapterX}

\section{\label{sec:intro}Introduction}

\subsection{\label{sec:intro.intro}Intro Intro (remove this subsection header later)}
Knowledge of the thermal conductivity of solids is key in a wide range of technological applications and for our understanding of natural systems. For example, in the Earth's lower mantle thermal conductivity controls the nature of planetary convection (\citet{Tosi2013}), and the heat flux out of the core which powers the geotherm. Low thermal conductivities are required in thermoelectric materials, to maximise the efficiency of heat-electricity conversion (\citet{Snyder2008}).

A range of atomic scale simulation methods are available to determine the lattice thermal conductivity of materials. These are invaluable for calculating thermal conductivity at conditions of which experiments are difficult, e.g. the extreme conditions found in the Earth's lower mantle (pressures and temperatures up to 136~GPa and 4000~K at the core-mantle boundary).

(MOVE - to where though?) Many studies assume lowermost mantle thermal conductivity to be 10~\wmk~(e.g. \citet{Lay2008}), but uncertainty in the extrapolation of results made at low pressures and temperatures gives a range of 4~-~16 \wmk~(\citet{Brown1986, Osako1991, Hofmeister1999, Goncharov2009, Manthilake2011, Ohta2012}).